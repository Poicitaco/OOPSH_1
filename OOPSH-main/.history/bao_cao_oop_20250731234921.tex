\documentclass[12pt,a4paper]{article}
\usepackage[utf8]{vietnam}
\usepackage{geometry}
\usepackage{graphicx}
\usepackage{float}
\usepackage{listings}
\usepackage{xcolor}
\usepackage{hyperref}
\usepackage{fancyhdr}
\usepackage{titlesec}
\usepackage{enumitem}
\usepackage{amsmath}
\usepackage{booktabs}
\usepackage{longtable}
\usepackage{array}
\usepackage{multirow}
\usepackage{wrapfig}
\usepackage{rotating}
\usepackage{caption}
\usepackage{subcaption}
\usepackage{setspace}

% Cấu hình trang
\geometry{
    top=2cm,
    left=3cm,
    bottom=2cm,
    right=2cm
}

% Cấu hình font
\renewcommand{\familydefault}{\rmdefault}
\renewcommand{\rmdefault}{ptm}

% Cấu hình màu cho code
\definecolor{codegreen}{rgb}{0,0.6,0}
\definecolor{codegray}{rgb}{0.5,0.5,0.5}
\definecolor{codepurple}{rgb}{0.58,0,0.82}
\definecolor{backcolour}{rgb}{0.95,0.95,0.92}

% Cấu hình listing cho code
\lstdefinestyle{mystyle}{
    backgroundcolor=\color{backcolour},   
    commentstyle=\color{codegreen},
    keywordstyle=\color{magenta},
    numberstyle=\tiny\color{codegray},
    stringstyle=\color{codepurple},
    basicstyle=\ttfamily\footnotesize,
    breakatwhitespace=false,         
    breaklines=true,                 
    captionpos=b,                    
    keepspaces=true,                 
    numbers=left,                    
    numbersep=5pt,                  
    showspaces=false,                
    showstringspaces=false,
    showtabs=false,                  
    tabsize=2
}

\lstset{style=mystyle}

% Cấu hình hyperref
\hypersetup{
    colorlinks=true,
    linkcolor=blue,
    filecolor=magenta,      
    urlcolor=cyan,
    pdftitle={Báo cáo OOP - Hệ thống Quản lý Kỳ thi Sát hạch},
    pdfpagemode=FullScreen,
}

% Cấu hình header và footer
\pagestyle{fancy}
\fancyhf{}
\fancyfoot[C]{\thepage}
\renewcommand{\headrulewidth}{0pt}

% Cấu hình section
\titleformat{\section}{\Large\bfseries}{\thesection}{1em}{}
\titleformat{\subsection}{\large\bfseries}{\thesubsection}{1em}{}
\titleformat{\subsubsection}{\normalsize\bfseries}{\thesubsubsection}{1em}{}

% Cấu hình line spacing
\onehalfspacing

\begin{document}

% Bìa chính
\begin{titlepage}
    \centering
    \vspace*{2cm}
    
    {\Large\bfseries TRƯỜNG ĐẠI HỌC CÔNG NGHỆ THÔNG TIN}\\
    \vspace{0.5cm}
    {\Large\bfseries ĐẠI HỌC QUỐC GIA THÀNH PHỐ HỒ CHÍ MINH}\\
    
    \vspace{2cm}
    
    {\Large\bfseries KHOA CÔNG NGHỆ THÔNG TIN}\\
    
    \vspace{3cm}
    
    {\Huge\bfseries BÁO CÁO CUỐI KỲ}\\
    \vspace{0.5cm}
    {\Large\bfseries MÔN: LẬP TRÌNH HƯỚNG ĐỐI TƯỢNG VỚI JAVA}\\
    
    \vspace{2cm}
    
    {\Huge\bfseries HỆ THỐNG QUẢN LÝ KỲ THI SÁT HẠCH}\\
    {\Large\bfseries (OOPSH - Driving License Examination System)}\\
    
    \vspace{3cm}
    
    \begin{tabular}{ll}
        \textbf{Giảng viên hướng dẫn:} & \textbf{Hà Thị Kim Dung} \\
        \textbf{Lớp:} & \textbf{OOP N02} \\
        \textbf{Học kỳ:} & \textbf{2024-2025} \\
    \end{tabular}
    
    \vspace{2cm}
    
    {\Large\bfseries Thành phố Hồ Chí Minh, 2025}
    
\end{titlepage}

% Bìa lót
\newpage
\begin{titlepage}
    \centering
    \vspace*{2cm}
    
    {\Large\bfseries TRƯỜNG ĐẠI HỌC CÔNG NGHỆ THÔNG TIN}\\
    \vspace{0.5cm}
    {\Large\bfseries ĐẠI HỌC QUỐC GIA THÀNH PHỐ HỒ CHÍ MINH}\\
    
    \vspace{2cm}
    
    {\Large\bfseries KHOA CÔNG NGHỆ THÔNG TIN}\\
    
    \vspace{3cm}
    
    {\Huge\bfseries BÁO CÁO CUỐI KỲ}\\
    \vspace{0.5cm}
    {\Large\bfseries MÔN: LẬP TRÌNH HƯỚNG ĐỐI TƯỢNG VỚI JAVA}\\
    
    \vspace{2cm}
    
    {\Huge\bfseries HỆ THỐNG QUẢN LÝ KỲ THI SÁT HẠCH}\\
    {\Large\bfseries (OOPSH - Driving License Examination System)}\\
    
    \vspace{3cm}
    
    \begin{tabular}{ll}
        \textbf{Giảng viên hướng dẫn:} & \textbf{Hà Thị Kim Dung} \\
        \textbf{Lớp:} & \textbf{OOP N02} \\
        \textbf{Học kỳ:} & \textbf{2024-2025} \\
    \end{tabular}
    
    \vspace{2cm}
    
    \begin{table}[h]
    \centering
    \begin{tabular}{|l|l|l|l|}
    \hline
    \textbf{STT} & \textbf{Họ và tên thành viên} & \textbf{Điểm bằng số} & \textbf{Điểm bằng chữ} \\
    \hline
    1 & [Tên thành viên 1] & & \\
    \hline
    2 & [Tên thành viên 2] & & \\
    \hline
    3 & [Tên thành viên 3] & & \\
    \hline
    \end{tabular}
    \end{table}
    
    \vspace{2cm}
    
    {\Large\bfseries Thành phố Hồ Chí Minh, 2025}
    
\end{titlepage}

% Mục lục
\tableofcontents
\newpage

\section{Giới thiệu về bài toán}

\subsection{Định nghĩa}

Hệ thống Quản lý Kỳ thi Sát hạch (OOPSH) là ứng dụng desktop JavaFX quản lý toàn diện các kỳ thi sát hạch bằng lái xe, hỗ trợ quản lý thí sinh, giám thị, lịch thi, kết quả và báo cáo.

\subsection{Ví dụ minh họa}

Hệ thống áp dụng trong trung tâm sát hạch, giúp quản lý thông tin thí sinh, lên lịch thi, chấm điểm, thống kê tỷ lệ đậu/rớt và xuất chứng chỉ.

\section{Phân tích bài toán}

\subsection{Mô tả hệ thống}

\subsubsection{Tổng quan hệ thống}

OOPSH sử dụng kiến trúc 3 tầng:
\begin{itemize}
    \item \textbf{Presentation Layer}: JavaFX với FXML
    \item \textbf{Business Logic Layer}: Controllers và Services  
    \item \textbf{Data Access Layer}: DAO classes và XML files
\end{itemize}

\subsubsection{Sơ đồ hệ thống}

\begin{figure}[H]
\centering
\includegraphics[width=0.8\textwidth]{images/system_architecture.png}
\caption{Sơ đồ kiến trúc hệ thống OOPSH}
\label{fig:system_architecture}
\end{figure}

\subsubsection{Phân quyền người dùng}

Hệ thống hỗ trợ 3 vai trò:
\begin{enumerate}
    \item \textbf{Admin}: Quản lý người dùng, loại thi, lịch thi, báo cáo
    \item \textbf{Examiner}: Xem thí sinh, chấm điểm, tạo báo cáo
    \item \textbf{Candidate}: Đăng ký thi, xem lịch, kết quả, thanh toán
\end{enumerate}

\subsection{Phân tích chức năng}

\subsubsection{Biểu đồ phân rã chức năng}

\begin{figure}[H]
\centering
\includegraphics[width=0.9\textwidth]{images/function_decomposition.png}
\caption{Biểu đồ phân rã chức năng hệ thống}
\label{fig:function_decomposition}
\end{figure}

\subsubsection{Chức năng chính}

\begin{enumerate}
    \item \textbf{Quản lý người dùng}: CRUD người dùng, phân quyền, tìm kiếm
    \item \textbf{Quản lý kỳ thi}: Tạo loại thi, lên lịch, phân công giám thị
    \item \textbf{Quản lý đăng ký}: Đăng ký thi, thanh toán, xác nhận
    \item \textbf{Quản lý kết quả}: Chấm điểm, tính tổng, xác định đậu/rớt
    \item \textbf{Thống kê báo cáo}: Tỷ lệ đậu/rớt, doanh thu, xuất báo cáo
\end{enumerate}

\subsection{Phân tích cơ sở dữ liệu}

\subsubsection{Cấu trúc dữ liệu}

Hệ thống sử dụng XML với các bảng chính:

\begin{table}[H]
\centering
\begin{tabular}{|l|l|l|}
\hline
\textbf{Bảng} & \textbf{Mô tả} & \textbf{Số trường} \\
\hline
Users & Thông tin người dùng & 15 \\
\hline
ExamTypes & Loại thi & 8 \\
\hline
ExamSchedules & Lịch thi & 9 \\
\hline
Registrations & Đăng ký thi & 7 \\
\hline
Results & Kết quả thi & 9 \\
\hline
Payments & Thanh toán & 8 \\
\hline
\end{tabular}
\caption{Cấu trúc các bảng dữ liệu}
\label{tab:database_structure}
\end{table}

\subsubsection{Sơ đồ quan hệ dữ liệu}

\begin{figure}[H]
\centering
\includegraphics[width=0.9\textwidth]{images/database_relationship.png}
\caption{Sơ đồ quan hệ dữ liệu}
\label{fig:database_relationship}
\end{figure}

\subsubsection{File XML mẫu}

\begin{lstlisting}[language=XML, caption=File users.xml mẫu]
<?xml version="1.0" encoding="UTF-8" standalone="no"?>
<users>
    <user>
        <id>1</id>
        <username>admin</username>
        <password>admin123</password>
        <role>ADMIN</role>
        <fullName>Quản trị viên</fullName>
        <email>admin@satheach.com</email>
        <createdDate>2025-07-31</createdDate>
        <status>ACTIVE</status>
    </user>
</users>
\end{lstlisting}

\subsection{Cài đặt và sử dụng}

\subsubsection{Yêu cầu hệ thống}

\begin{itemize}
    \item Java JDK 17+
    \item Apache NetBeans 23
    \item Maven 3.6+
    \item Windows/Linux/macOS
    \item 4GB RAM, 1GB ổ cứng
\end{itemize}

\subsubsection{Cài đặt}

\begin{enumerate}
    \item Cài đặt Java JDK 17, NetBeans 23, Maven 3.6+
    \item Clone project: \texttt{git clone <repository-url>}
    \item Build: \texttt{mvn clean compile}
    \item Chạy: \texttt{mvn javafx:run}
\end{enumerate}

\subsubsection{Hướng dẫn sử dụng}

\paragraph{Thông tin đăng nhập:}
\begin{itemize}
    \item \textbf{Admin}: admin / admin123
    \item \textbf{Examiner}: giamthi001 / gt123456  
    \item \textbf{Candidate}: thisinh001 / ts123456
\end{itemize}

\begin{figure}[H]
\centering
\includegraphics[width=0.7\textwidth]{images/login_screen.png}
\caption{Màn hình đăng nhập}
\label{fig:login_screen}
\end{figure}

\begin{figure}[H]
\centering
\includegraphics[width=0.9\textwidth]{images/admin_dashboard.png}
\caption{Dashboard Admin}
\label{fig:admin_dashboard}
\end{figure}

\subsubsection{Tính năng nổi bật}

\begin{enumerate}
    \item \textbf{Giao diện hiện đại}: Material Design, responsive, animation
    \item \textbf{Tìm kiếm nâng cao}: Gần đúng, khoảng điểm, ngày
    \item \textbf{Thống kê chi tiết}: Dashboard, biểu đồ, báo cáo
    \item \textbf{Validation tốt}: Input validation, error handling
\end{enumerate}

\section{Phân tích kỹ thuật}

\subsection{Kiến trúc hệ thống}

\subsubsection{Mô hình MVC}

\begin{figure}[H]
\centering
\includegraphics[width=0.8\textwidth]{images/mvc_architecture.png}
\caption{Mô hình MVC trong OOPSH}
\label{fig:mvc_architecture}
\end{figure}

\subsubsection{Pattern Design}

\begin{lstlisting}[language=Java, caption=DAO Pattern]
public abstract class BaseDAO<T, ID> implements CrudOperations<T, ID> {
    protected final String xmlFilePath;
    protected final ReadWriteLock lock = new ReentrantReadWriteLock();
    
    protected abstract String getElementName();
    protected abstract T elementToEntity(Element element);
    protected abstract Element entityToElement(Document doc, T entity);
}
\end{lstlisting}

\subsection{Xử lý dữ liệu XML}

\begin{lstlisting}[language=Java, caption=DOM Parser]
protected Document loadDocument() {
    lock.readLock().lock();
    try {
        DocumentBuilderFactory factory = DocumentBuilderFactory.newInstance();
        DocumentBuilder builder = factory.newDocumentBuilder();
        return builder.parse(new File(xmlFilePath));
    } finally {
        lock.readLock().unlock();
    }
}
\end{lstlisting}

\section{Đánh giá và kết luận}

\subsection{Đánh giá hệ thống}

\subsubsection{Ưu điểm}

\begin{enumerate}
    \item \textbf{Đáp ứng đầy đủ yêu cầu đề bài}: XML, JavaFX, tìm kiếm, thống kê
    \item \textbf{Kiến trúc tốt}: MVC, design patterns, code rõ ràng
    \item \textbf{Giao diện hiện đại}: Material Design, thân thiện
    \item \textbf{Tính năng phong phú}: 3 vai trò, quản lý toàn diện
\end{enumerate}

\subsubsection{Hạn chế}

\begin{enumerate}
    \item \textbf{Hiệu suất}: XML không phù hợp dữ liệu lớn
    \item \textbf{Bảo mật}: Dữ liệu chưa mã hóa, audit trail hạn chế
    \item \textbf{Tính mở rộng}: Khó scale, concurrent access hạn chế
\end{enumerate}

\subsection{Kết luận}

OOPSH đã phát triển thành công với:
\begin{itemize}
    \item Đáp ứng đầy đủ yêu cầu đề bài
    \item Kiến trúc MVC tốt, design patterns phù hợp
    \item Giao diện hiện đại, tính năng phong phú
    \item Sẵn sàng triển khai thực tế
\end{itemize}

\subsection{Hướng phát triển}

\begin{enumerate}
    \item Nâng cấp database (MySQL/PostgreSQL)
    \item Tăng cường bảo mật
    \item Phát triển web application
    \item Mobile app
    \item AI/ML integration
\end{enumerate}

\section{Tài liệu tham khảo}

\begin{enumerate}
    \item Oracle Corporation. (2023). \textit{JavaFX Documentation}.
    \item Apache Software Foundation. (2023). \textit{Maven Documentation}.
    \item Material Design Team. (2023). \textit{Material Design Guidelines}.
    \item Freeman, E., et al. (2004). \textit{Head First Design Patterns}.
    \item Gamma, E., et al. (1994). \textit{Design Patterns}.
\end{enumerate}

\end{document} 